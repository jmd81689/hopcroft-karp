%****************************************************************************
%** Copyright 2001, 2002, 2003, 2004 by Lukas Ruf, <lukas.ruf@lpr.ch>
%** Information is provided under the terms of the
%** GNU Free Documentation License <http://www.gnu.org/copyleft/fdl.html>
%** Fairness: Cite the source of information, visit <http://www.topsy.net>
%****************************************************************************
%** Last Modification: 2005-07-11 1600
%** 2004-02-17: Lukas Ruf
%**             Added recommendation by Thomas Duebendorfer
%**             Added different babel languages
%**             Added more comments
%** 2004-10-16: Lukas Ruf
%**             More comments
%**             Added subfigure
%** 2005-07-11	Bernhard Tellenbach
%**							Added \abbrev command to generate a list of abbrevations
%**							Removed support for psfig and epsfig (old)
%** 						Adapted syntax for new nomencl packet version
%****************************************************************************

\RequirePackage{times}

\usepackage[english]{babel}
%-% \usepackage[german]{babel}
%-% \usepackage[ngerman]{babel}

\usepackage[latin1]{inputenc}
\usepackage[T1]{fontenc}
\usepackage{type1cm}

\usepackage{a4}

\usepackage[dvips]{graphicx}
\graphicspath{{Figures/},{Pictures/}}
\usepackage{subfigure}

\usepackage{fancyhdr}
\usepackage{fancybox}

\usepackage{float}
\usepackage{longtable}
\usepackage{paralist}
\usepackage{url}
\usepackage{portland}
\usepackage{lscape}
\usepackage{moreverb}

\usepackage{nomencl}
  \let\abbrev\nomenclature
  \renewcommand{\nomname}{List of Abbrevations}
  \setlength{\nomlabelwidth}{.25\hsize}
  \renewcommand{\nomlabel}[1]{#1 \dotfill}
  \setlength{\nomitemsep}{-\parsep}
  %For old nomencl package, uncomment this:
  \makeglossary 
  %For new nomencl package, uncomment this:
  %\makenomenclature

\usepackage[normalem]{ulem}
  \newcommand{\markup}[1]{\uline{#1}}
  
   
%% Thanks to Thomas Duebendorfer: Should create smoother fonts
\usepackage{ae,aecompl}


\usepackage[first,bottomafter,light,dvips]{draftcopy}
\draftcopyName{Draft v0.1}{120}

%\addtolength{\textwidth}{2cm}
%\addtolength{\textheight}{2cm}
%\addtolength{\oddsidemargin}{-1.0cm}
%\addtolength{\evensidemargin}{-1.0cm}
%addtolength{\topmargin}{-1.5cm}

%% No Serifs: Put comment markers in front of the next three lines otherwise
%\renewcommand{\ttdefault}{cmtt}
%\renewcommand{\rmdefault}{phv}  % Helvetica for roman type as well as sf
%\renewcommand{\ttdefault}{pcr}  % use Courier for fixed pitch, if needed

\newcommand{\?}{\discretionary{/}{}{/}}
\newcommand{\liter}[0]{/home/ruf/Lib/Bibl/}
\newcommand{\fref}[1]{\mbox{Figur~\ref{#1}}}

\pagestyle{fancy}
%%-lpr Note: 'chapters' are defined for 'book's only
%%-lpr       in articles, we make use of sections only
%%-lpr \renewcommand{\chaptermark}[1]{\markboth{#1}{}}
\renewcommand{\sectionmark}[1]{\markright{\thesection\ #1}}
\fancyhf{}
\fancyhead[LE,RO]{\bfseries\thepage}
\fancyhead[LO]{\bfseries\rightmark}
\fancyhead[RE]{\bfseries\leftmark}
\renewcommand{\headrulewidth}{0.5pt}
\addtolength{\headheight}{0.5pt}
\fancypagestyle{plain}{%
   \fancyhf{}
   \fancyfoot[C]{\bfseries \thepage}
   \fancyhead{}%get rid of headers on plain pages
   \renewcommand{\headrulewidth}{0pt} % an the line
}
\newcommand{\clearemptydoublepage}{\newpage{\pagestyle{empty}\cleardoublepage}}

%\setlength{\parindent}{0in}

\hyphenation{Lukas not-to-hyphen-else-where}

\newcommand{\Appendix}[2][?]
{
  \refstepcounter{section}
  \addcontentsline{toc}{appendix}
  {
    \protect\numberline{\appendixname~\thesection} %1
  }
  {
    \flushright\large\bfseries\appendixname\ \thesection\par
    \nohypens\centering#1\par
  }
  \vspace{\baselineskip}
}

\let\margin\marginpar
\newcommand\myMargin[1]{\margin{\raggedright\scriptsize #1}}
\renewcommand{\marginpar}[1]{\myMargin{#1}}

\newcommand\CHECK{\myMargin{CHECK}}
\newcommand\NEW{\myMargin{NEW}}

